\documentclass{scrartcl}

\usepackage{listings}
\usepackage{amsmath}

\title{Design of Embedded System\\Exercise 4}
\author{Erin van der Veen - s4431200\\
	Brigel Pineti - s1005549}

\begin{document}
\maketitle

\section*{4a}
Since all tasks wait for the mysync semaphore, and task 2 has the highest priority, followed by task 1, and then task 0, the order of execution of the tasks is the same.
First task 2 is executed, followed by 1 and then 0.

\section*{4b}
Now that task 1 and task 2 have the same priority, and the semaphore is set to \lstinline|S_FIFO| these tasks are executed in the other of which they wait for the semaphore.
Since task 1 is created before task 2 is created, it is first.
Task two is executed second, followed by task 0.

Look at it this way, the scheduling ensures that tasks with the highest priority are executed first.
However, when two tasks have the same priority, they are executed in the order in which they entered the queue of the semaphore.

\section*{4c}
First, task 2 runs until it reaches it's halfway point.
At that point, it increases the priority of Task 1 to be higher that it.
This means, that the CPU will switch to task 1, which is executed until completed (since it has the highest priority).
After it has completed, task 2 is again the task with the highest priority.
However, the first thing task 2 does, is increase the priority of task 0 to be higher than itself.
The CPU therefore, executes and completes execution of task 0.
When task 0 has completed, task 2 once again becomes the task with the highest priority, and is executed until completed.

\section*{4d}
Task 2 is executed first since it has the highest priority (52).
When it reaches its halfway point, it changes priority to 38.
Task 1 then takes over, since it has the highest priority (51).
When task 1 reaches its halfway point, it changes its priority to 39.
Task 0 then takes over with a priority of 50.
When task 0 reaches its halfway point it sets its own priority to 40.
Since task 0 still has the highest priority (40, over 39 and 38) it continues until completion.
Task 1 then has the highest priority (of 39), so it runs until completed.
The last task is task 2, it runs after task 1 has completed.

\end{document}
