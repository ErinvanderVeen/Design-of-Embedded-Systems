\documentclass{scrartcl}

\usepackage{listings}
\usepackage{graphicx}
\usepackage{amsmath}
\usepackage{caption}
\newenvironment{longlisting}{\captionsetup{type=listing}}{}
\usepackage{minted}
\setminted{fontsize=\small,breaklines,breakanywhere,tabsize=4}

\title{Design of Embedded Systems\\Exercise 9\\Using Clean on a Real Time Operating System}
\author{Erin van der Veen - s4431200}

\begin{document}
\maketitle

\tableofcontents
\pagebreak

\section{The Compiler}
\subsection{The Procedure}
Compiling a Clean program is done in three distinct steps.
First, the Clean code (.icl and .dcl) is compiled to an intermidiate language called ``Core Clean''.
This step is performed by a program that was written in Clean.
Next, ``Core Clean'' is compiled to another, platform independent, intermediate language (.abc).
This step is performed by a mixture of programs written in both Clean and C.
In the last step, abc-code is compiled to bytecode (.o), which is done entirely by C-code.

\subsection{Bootstrapping the Compiler}
The Clean compiler can, in general, be downloaded from the ftp server\footnote{ftp.cs.ru.nl/pub/Clean/builds/}.
Unfortunately, there is no build-script available for ARM\footnote{gitlab.science.ru.nl/clean-and-itasks/clean-build}.
As such, we must build the environment and compiler ourselves.
Getting the source-code of the compiler and trying to just build it is impossible since it requires that the compiler is already installed.
The last step of the compilation procedure does not require Clean, however.
Using this knowledge, the compiler can be bootstrapped using another machine to generate the .abc code, and then compiled to a native binary using a C program.

\section{Linking Clean and C}
\subsection{Clean to C}
\begin{longlisting}
	\begin{minted}{Clean}
		usleep :: !Int !*World -> (!Int, !*World)
		usleep i w = code {
			ccall usleep "I:I:A"
		}
	\end{minted}
	\caption{Calling the usleep C function from Clean\label{L:ctoclean_example}}
\end{longlisting}
Calling C functions from Clean is relatively easy.
Consider code in Listing \ref{L:ctoclean_example}.
The \mintinline{Clean}{code} keyword tells the compiler that whatever is in between the brackets should be interpreted directly as abc code.
In this abc-block we then use the \mintinline{Clean}{ccall} to tell the compiler that we want to call a C-function.
\mintinline{Clean}{ccall} takes 2 arguments, the first is the name of the C-function that we want to call.
The second is a description of the arguments and return values of the function that we call.
This description consists of 3 parts (``x:y:z'').
`x' denotes the arguments that must be passed to the function, in this case, we sent a single Integer.
The return value is denoted by y, in the example, we expect the function to return an Integer.
The last part (z) denotes any arguments that must not be passed to the function.
More often than not, you will see that this is the *World, which ensures that the language stays pure.
Linking is done by passing \mintinline{Bash}{-l -lib} to \mintinline{Bash}{clm}.

\subsection{C to Clean}
Clean offers functionality to export its functions with a foreign function interface.
This is done using the by adding \mintinline{Clean}{foreign export ccall hello_world} to a .icl file.
The function must also be exported in the .dcl file.
It must be noted, however, that Clean uses a run time system.

\section{The Runtime System}
Clean is a language that makes use of a garbage collector and several other implementation details that require a run time system.
Unlike Java and C\#, this run time system is linked into final binary.
Therefore, running any Clean program is preceded by that init system of the run time system.
I noted before that it is possible to call Clean functions from C, while this is true, the run time system must be initialized before C can call a Clean function.
With the help of Drs. van Groningen, I am currently trying to create a modified version of the run time system\footnote{github.com/ErinvanderVeen/C-to-Clean}, that allows you to initialize it from C.\footnote{UPDATE: This method is working since 2017-10-07 but was not used due to the fact that it would require a complete rewrite of the program}
This would mean that it would no longer be necesaary to start the program in Clean.

\section{Xenomai Tasks and Threads}
\subsection{Xenomai Tasks}
A Xenomai task is a feature of the Alchemy API that allows a process to create tasks that run on the Cobalt kernel.
The Alchemy API also allows additional features like periodic tasks and semaphores, but for the sake of simplicity, these will be disregarded for the rest of this assignment.
In the background, when the Alchemy API receives a call to create a new task, it does some preprocessing, and then uses the Cobalt API to create a xthread.
An xthread is a thread that runs on the Cobalt kernel.
One of the parameters to spawn an xthread is a function pointer, this is the function that is executed by the xthread.
Running on the Cobalt kernel ensures that the thread has priority over any thread that runs on the normal Linux kernel, and is desired in real time applications.

\subsection{Xenomai Tasks in Clean}
\begin{longlisting}
	\begin{minted}{Clean}
		thunk_to_module_name_pointer :: !(a a->Bool) -> Int;
		thunk_to_module_name_pointer a = code {
			pushD_a 0
			pop_a 1
			push_b 0
			load_si16 0
			addI
			load_i 6
		}
	\end{minted}
	\caption{G$\forall$ST using abc code to get a module name\label{L:abc_m_p}}
\end{longlisting}
As mentioned in the subsection above, a function pointer must be provided to the Alchemy API in order for it to start the task.
The G$\forall$ST framework shows that it might be possible to do this in Clean by using the \mintinline{Clean}{code}-block mentioned earlier.
Consider, for example the code in Listing \ref{L:abc_m_p}, where they manage to extract the name of a module.
Unfortunately, figuring out how to obtain the function pointer using a similar method was outside the scope of this assignment.
Alternatively, I created a C-function that simply returns the function pointer.

\subsubsection{Starting the Task}
Clean does not support multi-threading, although there has been some research that looks into this\footnote{ru.nl/publish/pages/769526/z09\_jan\_groothuijse.pdf}.
This implies that a Clean program does not create the required context to allow for contextswitching within threads.
The initial context switch that takes place when Clean creates a Xenomai Task happens without any issues (since the new thread is created by the Xenomai API).
This context switch that takes place whenever the Xenomai Task is blocked, however, causes a segmentation fault.
This is due to the context that I mentioned earlier and cannot be helped without either:
\begin{enumerate}
	\item Not starting the process as a Clean process
	\item Implementing multi threading in Clean
\end{enumerate}
As mentioned before, both of these are as of yet outside the scope of this assigment.

\section{Real Time Behaviour in a Non Real Time Thread}
As mentioed above, it is (almost) impossible to run a real time thread from Clean.
This does not prevent us from trying to emulate the behaviour of a real time thread however.
Calling any \mintinline{C}{rt\_} function schedules the current thread on the Cobalt kernel.
This allows us to reliably spin for a certain amount of time, without the possible delay of a normal kernel.

Unfortunately the Alchemy API does not provide an interface to do a blocking read on the Cobalt kernel.

\section{RTIME}
The RTIME in the Alchemy API is a type alias for the unsigned long long.
On a Raspberry Pi 2 (the one I use) that type is stored in 64 bits.
The maximum that Clean can index on a 32 bit system (like the Raspberry Pi 2) is the Int (of 32 bits).
In order to be able to store unsigned long long in Clean, I used a Int, Int tuple.
Where the idea was to store the most significant bits in the left integer, and the least significant bits in the right integer.
In order parse the RTIME that the Alchemy API returns, I used the derefInt function in Clean.
I would first read 4 bytes and store those, then increment the pointer by 4, and finally read the last 4 bytes.
Unfortunately, I ran into the problem that the Raspberry Pi stores it's data in a little Endian format.
After careful study of the code, I was not sure if \mintinline{Clean}{System._Pointer} takes the Endian of an architecture into consideration.
As such, I decided it was not within the scope of this assigment to figure this out.
Instead I created a type synonym \mintinline{Clean}{:: RTIME :== Pointer} that would be used in the calculations.
I overloaded this datatype in both $<$ RTIME and $-$ RTIME and used ccalls to calculate the results.

\section{Implementation}
\subsection{Direction}
There are two possible locations of the pendulum; it is either to the left of the lichtsensor, or to the right.
In the initialization stage of the process the current location of the pendulum is determined by getting the system time of three interruptions.
If the t1 - t2 is smaller than t3 - t2 the pendulum must be on the LEFT side of the pendulum after t3.
In the other case, it must be on the RIGHT side of the sensor.

In Clean, I implemented this using an Algebraic DataType: \mintinline{Clean}{:: Direction = LEFT | RIGHT}.
After that, the code always waits until the pendulum passes the sensor twice, allowing us to assume that the sensor is always on the right side.

\subsection{Period}
Unfortunately, due to the limitations mentioned above, the period cannot be completely calculated in Clean.
Most of it, however, can.
First, 8 bytes of space is allocated on the heap.
After that, the process waits until the pendulum passes the sensor going right.
The time is stored, and we then wait for it to pass going left.
This time is then stored, and the difference is returned as the period.

\subsection{LEDs}
While the initialization (getting the filedescriptors) could be done in Clean, it would not have any advantage over just doing it in C.
As such, the LEDs are initialized in C, and the filediscriptors are then passed to Clean.
Initialization is done by the following function: \mintinline{Clean}{setupLEDS :: !*World -> (!Int, !*World)}.
When calling a C-function from Clean, the C funtion must always return some value.
Right now, the C-function always returns a 1. Ideally, however, the function would return any possible error code.

Retrieving the file descriptors is done by mapping a function that retrieves the file descriptor given some LED Identifier: \mintinline{Clean}{setupLEDS :: !*World -> (!Int, !*World)}.
Given that the *World must be passed, a mapSt must be used: \mintinline{Clean}{mapSt getFD leds world}.

In order to distinguish between normal Integers and Integers that identify the LED file descriptors, I use a type synonym: \mintinline{Clean}{:: LED :== Integer}.

\section{Acknowledgments}
A special thanks to Camil Staps for helping me set up the compiler, and any other help he provided me.

\end{document}
