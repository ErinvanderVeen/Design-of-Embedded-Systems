\documentclass{scrartcl}

\usepackage{listings}
\usepackage{amsmath}

\title{Design of Embedded System\\Exercise 3}
\author{Erin van der Veen - s4431200\\
	Brigel Pineti - s1005549}

\begin{document}
\maketitle

\section*{3a}
The global counter is first incremented to 10 by taskOne, after which it is decremented to 0 by taskTwo.

\section*{3b}
Access to the global variable is handled through a semaphore.
A task needs to wait until the variable is released to inc/dec global.
Since we use S\_FIFO as the semaphore behaviour, the task alterate in their execution.

In order to implement the bonus, two semaphores can be created.
The first will determine whether a task can INC the variable, the other will determine whether a task can DEC the variable.
The tasks then increment the semaphore of the opposite task, to allow one of the other tasks to perform their operation.

\section*{3c}
After the tasks have been created, the first thing they do is wait on a semaphore.
When all tasks are created, the main then broadcasts the semaphore, to allow all blocked tasks to execute.
In order to ensure that the tasks are executed in order of priority, the semaphore is in S\_PRIO mode.

The same behaviour can also be achieved by using \lstinline|rt_sem_v|.
This is achieved by simply calling this function n times, where n is the amount of tasks that were created.

\end{document}
