\documentclass{scrartcl}

\usepackage{listings}
\usepackage{amsmath}

\title{Design of Embedded System\\Exercise 6}
\author{Erin van der Veen - s4431200\\
	Brigel Pineti - s1005549}

\begin{document}
\maketitle

In order to start Medium task after the Low task has already started we can do two things.
\begin{enumerate}
	\item Return to the normal kernel by sleeping the Low task
	\item Start the Medium task from the Low task
\end{enumerate}
In the code, I ended up using the second method, as it provides more deterministic behaviour.
This is of course only done after Low task has already locked the semaphore.
The medium task then prints its string ``Medium task running'' twice, before starting High task.
High task tries to lock the semaphore, but is forced to wait.
This leads to a context switch to Medium task, which then runs until completion.
Since High tasked is still waiting for the semaphore, a context switch takes place to Low task.
The first thing Low task does is unlock the semaphore.
This removes High task from the FIFO queue, and a context switch takes place to High task.
High tasks runs until completed, after which Low task does the same.

\end{document}
