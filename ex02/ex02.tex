\documentclass{scrartcl}

\usepackage{listings}
\usepackage{amsmath}

\title{Design of Embedded System\\Exercise 2}
\author{Erin van der Veen - s4431200\\
	Brigel LASTNAME - SNUMBER}

\begin{document}
\maketitle

\section*{2a}
Every task is given a unique name, this name is subsequently printed by the tasks as ``\lstinline|Task name: NAME|''.
Where NAME is the name given to the task.
The task with name ``Task 1'' is executed first, followed by ``Task 2'' etc.

\section*{2b}
Every task is now provided with both a unique name and a unique index.
Both are printed by the task in the following way:
\begin{lstlisting}
Task name: NAME
Task Nr: INDEX
\end{lstlisting}
Where NAME is once again the name of the task, and INDEX is the index that is given as an argument.
The task with name ``Task 1'' and index 1 is executed first, the rest are then printed in sequential order.

\section*{2c}
The tasks are still executed in the same order as in ex02a and ex02b.
This is most likely due to the very short execution time of the tasks.
The tasks probably complete their instructions before the next task is started, or before a context switch can take place.

\section*{2d}
Suppose we start a counter that counts in seconds the moment the process is executed.
Whenever this counter reaches a number of seconds that is $0 \mod 1$ (i.e. every second), Task 1 prints its index.
Whenever this counter reaches a number of seconds that is $0 \mod 2$ (i.e. every 2 seconds), Task 2 prints its index.
Whenever this counter reaches a number of seconds that is $0 \mod 3$ (i.e. every 3 seconds), Task 3 prints its index.

Given the way the code is written, this can easily be extended to work with up to $2^{16} - 1$ tasks.

\end{document}
