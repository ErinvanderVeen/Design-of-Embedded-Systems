\documentclass{scrartcl}

\usepackage{listings}
\usepackage{amsmath}

\title{Design of Embedded System\\Exercise 5}
\author{Erin van der Veen - s4431200\\
	Brigel Pineti - s1005549}

\begin{document}
\maketitle

\section*{5a}
As each task is created, they wait on the FIFO semaphore.
Then, when all tasks are created they are released in the order in which they were created.
Since no Round Robin Scheduling is used, the tasks are ran until completion, at which point the next task is run.
\begin{enumerate}
	\item Task 0 starts
	\item Task 0 completes
	\item Task 1 starts
	\item Task 1 completes
	\item Task 2 starts
	\item Task 2 completes
\end{enumerate}

\section*{5b}
The output can either be the same as before, or it can be different.
If you set the timeslice of the tasks to be the same or higher than the EXECTIME, nothing changes.
When, however, you set it to equal the SPINTIME, the tasks will take turns at printing their statements. I.e.
\begin{enumerate}
	\item \dots
	\item Task 0 prints
	\item Task 1 prints
	\item Task 2 prints
	\item \dots
\end{enumerate}

\section*{5c}
Given that task 3 has a higher priority than the other tasks, it is always allowed to complete first.
After that however, the Round Robin Scheduling has the same impact as it had in 5b.
\begin{enumerate}
	\item Task 3 starts
	\item Task 3 completes
	\item \dots
	\item Task 0 prints
	\item Task 1 prints
	\item Task 2 prints
	\item \dots
\end{enumerate}

\end{document}
