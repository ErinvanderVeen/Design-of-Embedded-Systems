\documentclass{scrartcl}

\usepackage{longtable}

\title{Mars Rover}
\subtitle{Requirements, Configuration and Development Plan}
\author{Erin van der Veen - s4431200\\
	Brigel Pineti - s1005549}

\begin{document}
\maketitle

\section{Requirements}
Table~\ref{tab:requirements} shows the individual requirements, including their priority and a short description.
\begin{longtable}{|l|p{1.7cm}|p{5cm}|p{5cm}|}
	\hline
	Code & Name & Description & Ideas on Implementation \\\hline
	MH1 & Backwards of Table & At no instance should the Robot Drive backwards of the table & The Ultrasonic Sensor in the back of the robot seems like the best candidate to determine if the Rover is about to drive of the table. \\\hline
	MH2 & Stay within White Border & At no Point should the robot move outside the white border. & This requirement has slight overlap with MH1. The color sensors can be be used to measure the white border, after which a turn can be made based on which of the sensors red the white color.\\\hline
	MH3 & Avoid Lakes & The Rover should never drive into one of the ``lakes''. & For forward movement, this can be sensed using the color sensors (every lake has a distinct color), for backwards movement the behavior from MH1 can be used.\\\hline
	MH4 & Avoid Rocks & The Rover should not run into rocks, unless with the explicit purpose of pushing the rock out of the field & \\\hline
	MH5 & Subsump-tion Architecture & Currently, the behavior framework waits until a behavior is completed before it assesses which behavior should be ``executed'' next. Ideally, a behavior should be terminated as soon as another behavior of higher priority indicates that is much be executed. & Two threads will be created, one that will deal with the execution of the behaviors, one that will constantly assess which behavior must run. \\\hline
	SH1 & Detect lakes & The rover should be able to at least find all lakes. & This does not necessarily mean that is must ``measure'' every lake, rather, that it should find all colors. \\\hline
	SH2 & Measure lakes & The rover should be able to measure each lake once it has found one, it should try to avoid measuring the same lake twice. & A variable system must be implemented to store the knowledge base of the Rover.\\\hline
	SH3 & Measure Rocks & Once a rock is found, the robot should be able to ``measure'' this rock. & \\\hline
	CH1 & Clean up Rocks & Once a rock is ``measured'', the rover can push this rock out of the field & \\\hline
	CH2 & Play Sound & The Rover could be able to play a sound whenever necessary & This must be implemented in both the DSL and C++ \\\hline
	WH1 & Path-finding & The Robot will not be able to look ahead, or have a internal representation of the field & \\\hline
\caption{Requirements of the Mars Rover DSL and Implementation, the code of a given requirement also indicates its priority. With MH = Must Have, SH = Should Have, CH = Could Have, WH = Wont Have}
\label{tab:requirements}
\end{longtable}

\section{Configuration Proposal}
Table~\ref{tab:config} shows the proposed configuration in the way is was given in the slides.
The decision was based on the core concept that Brick 1 is the master, and Brick 2 is the slave.
This implies that all systems (sensors and motors) that require fast interaction are connected to Brick 1, all other systems are connected to Brick 2.
It would be ideal if at least one color sensor (the center one) could also be connected to Brick 1, since it can be used to determine if the Robot is about to drive of the table.
We argue, however, that the separation of concerns (all similar sensors should be connected to one brick) is more important than connected this sensor to Brick 1.
The separation of concerns makes the code that will be generated by the DSL potentially much more readable.
\begin{table}
	\centering
	\begin{tabular}{|l|p{5cm}|p{5cm}|}
		\hline
		& EV3 Brick 1 & EV3 Brick 2 \\\hline
		Actuators & Left Motor \newline Right Motor & Measurement Motor \\\hline
		Sensors & Ultrasonic \{Front, Rear\} \newline Touch \{Left, Right\} & Color \{Left, Mid, Right\} \newline Gyro \\\hline
	\end{tabular}
	\caption{The Configuration of the two bricks as proposed}
	\label{tab:config}
\end{table}

\section{Development Plan}
Table~\ref{tab:plan} shows the development plan in its most pessimistic form.
\begin{table}[hb]
	\centering
	\begin{tabular}{|l|l|l|}
		\hline
		Week nr & Deadline & Plan \\\hline
		Week 0 & 28 Nov & Requirements, Configuration, Plan \\\hline
		Week 1 & 5 Dec & MH5, MH1 \\\hline
		Week 2 & 12 Dec & MH2 MH3, MH4 \\\hline
		Week 3 & 19 Dec & SH1, SH2, SH3 \\\hline
		Week 4 & 26 Dec & CH1, CH2, and Extra \\\hline
	\end{tabular}
	\caption{The Development Schedule}
	\label{tab:plan}
\end{table}
\end{document}
