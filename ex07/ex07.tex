\documentclass{scrartcl}

\usepackage{listings}
\usepackage{amsmath}

\title{Design of Embedded System\\Exercise 7}
\author{Erin van der Veen - s4431200\\
	Brigel Pineti - s1005549}

\begin{document}
\maketitle

\section*{7a}
7a is a very simple program, it simply sleeps for half a second, and then writes value to the pin.
In order to achieve an alternating effect, \lstinline|value| is then XOR'ed with 1.
This flips the value from 1 to 0, and 0 to 1.

\section*{7b}
7b simply extends the program of 7a with a task (``Peter'') that listens for Rising and Falling edges on the interrupt pin.
Whenever this happens, a global counter is increased by 1 and the value is printed.
The task also prints whether the interrupt occurred on a Rising edge (1), or a Falling edge (0).

\section*{7c}
7c has two possible interpretations.
\begin{enumerate}
	\item A single press should turn the LED on/off
	\item The LED should be lit for as long as the button is pressed
\end{enumerate}
In the implementation that we handed in, we follow definition 1.
Changing the code to definition 2, however, is done by simply chancing a single line such that ``Peter'' not only listens to Rising edges, but also to Falling edges.
Whenever a interrupt occurs, the amount of interrupts is printed, the LED is switched on/off, and the internal LED state of the program is XOR'ed.

\end{document}
